%% *************************************************************************
%%
%% This is a derivative work of the RIT Space Exploration Standard defining 
%% guidelines for content and formatting of project design documents.
%%
%% This document uses IEEEtran.cls, the official IEEE LaTeX class
%% for authors of the Institute of Electrical and Electronics Engineers
%% (IEEE) Transactions journals and conferences.
%%
%% *************************************************************************

%% *************************************************************************
% LaTeX REFERENCES
% ----------------
%   Intro to LaTeX: http://www.rpi.edu/dept/arc/docs/latex/latex-intro.pdf
%   Comprehensive LaTeX symbol list: http://tug.ctan.org/info/symbols/comprehensive/symbols-a4.pdf
%% *************************************************************************

% tell \LaTeX what kind of formatting to use
\documentclass[conference]{IEEEtran} % http://www.ctan.org/pkg/ieeetran
\usepackage{blindtext} % enable placeholder text generator
\usepackage{graphicx} % enable toolbox for embedding figures and pictures
\usepackage{nomencl} % enable package for adding a list of variables and constants at the beginning, aka "nomenclature"
\usepackage{siunitx} % enable package for easily formatting units
\usepackage{hyperref} % enable package for cross-referencing figures, sections, references etc.
% how to use hyperref: http://www2.washjeff.edu/users/rhigginbottom/latex/resources/lecture09.pdf
\usepackage[T1]{fontenc} % change text encoding to make it more crisp
\usepackage{etoolbox} % enable conditionals for help text
\usepackage{booktabs} % make beautiful tables!

% initialize nomenclature package
\makenomenclature{}

% set title. choose something as descriptive and precise as possible. Descriptive > sounding cool. remember this!
\title{Astrodynamics Team Fall Projects 2019 }


\author{
  % List the authors of the design document. The Champion should go first.
  % The \$~\$ markers tell \LaTeX{} to treat the text inside to be treated as a math expression. This way you can use operators like \textcaret{} to place characters as superscripts.
  % Some \LaTeX{} templates handle the author block in different ways. For example, the \href{http://www.worldscientific.com/worldscinet/jai}{Journal of Astronomical Instrumentation} requires the authors' addresses and emails to be included as well.
  % The \textbackslash{}thanks command puts the contents inside those brackets in a footnote at the bottom of the first page. Technically speaking, \textbackslash{}thanks is just a specially formatted footnote.
  % IEEE also has a ``long form'' author block for many authors. Check here for more information:
  % \url{https://tex.stackexchange.com/questions/156523/multiple-authors-with-common-affiliations-in-ieeetran-conference-template}
  % Read here for a more advanced options to modifying footnotes in the author block:  \url{http://tex.stackexchange.com/questions/826/symbols-instead-of-numbers-as-footnote-markers}
  %   Here, we use the IEEE long-form author block.
  \IEEEauthorblockN{% This block is for author Names.
    Amber Dubill\IEEEauthorrefmark{1},  %the number in the bracket is a reference number to identify this footnote. \LaTeX will figure out what symbol to put there.
    Evan Putnam\IEEEauthorrefmark{2},
    Dr. Jennifer Connelly\IEEEauthorrefmark{3}, and 
   Dr. Micheal Richmond\IEEEauthorrefmark{4},
  }
  \IEEEauthorblockA{% This block is for the author Affiliations, aka department and university
    RIT Space Exploration, Rochester Institute of Technology \\ %\\ starts a new line
    Rochester, N.Y. \\
    Email:
    \IEEEauthorrefmark{1}ald4035@rit.edu,
    \IEEEauthorrefmark{2}ald4035@rit.edu,
    \IEEEauthorrefmark{3}jlcsps@rit.edu,
    \IEEEauthorrefmark{4}mwrsps@rit.edu,
  }
  %%   Below, we use the short-form author block and basically hack it to suit our needs.
  % Philip~Linden$^{*\dagger}$%
  %   \thanks{$^{*}$Project Champion}%
  %   \thanks{$^{\dagger}$BS/MEng '17, Mechanical Engineering},
  % Austin~Bodzas$^{\ddagger}$%
  %   \thanks{$^{\ddagger}$BS '19, Computer Science},
  % Drew~Walters$^{\S}$%
  %   \thanks{$^{\S}$BS '18, Mechanical Engineering Technology},
  % T.J.~Tarazevits$^{**}$%
  %   \thanks{$^{**}$BS '19, Game Design \& Development}%

  %%   If there are many authors, consider using symbolic, numeric (aka arabic),  alphabet footnotes or a combination thereof.
  %% the recommended order for symbolic footnotes is
  %%   (1) asterisk        *   *
  %%   (2) dagger          †   \dagger
  %%   (3) double dagger   ‡   \ddagger
  %%   (4) section symbol  §   \S
  %%   et cetera. For higher counts, use 2x symbols (1)-(4) (i.e. (5) two asterisks **). Keep cycling through (1)-(4) using 3x, 4x, and so on.
  %%   Note that these symbol codes work in math mode and text mode.
  %%   There are ways to make LaTeX do this for you, but it is more advanced and not entirely necessary, especially for short author lists. Not worth the hassle, in my opinion.
}
% page header for pages other than cover page
\markboth{Project Design Document Standard}%
{Linden \MakeLowercase{\textit{et al.}}: RIT Space Exploration}

% Initial setup is over, start building the document itself
\begin{document}
\maketitle%
% correct bad hyphenation here, separated by spaces
\hyphenation{explor-ation}

\begin{abstract}
   The astrodynamics team of RIT Space Exploration (SPEX) has historically worked on engineering based projects and will continue to do so, but now will expand its projects for students interested in physics and astronomy. There is also plethora of smaller projects that would greatly improve the ability of the observatory while providing engineering experience to students in the realm of control systems, opto-mechanics, and instruments. This would also allow students to become familiar with the various equipment the observatory has to offer. The biggest task is a continuation of last year's rolloff dome motor project, considered one of the main objectives of this semester. The other higher level objective is a new computer vision algorithm that allows for identification of star patterns for telescope calibration. A freshman project to repair last year's radio telescope project is also proposed.
   
   
  % The RIT observatory currently does its best to host numerous students with the limited faculty and resources they possess. There is a plethora of smaller projects that would greatly improve the ability of the observatory while providing engineering experience to students in the realm of control systems, opto-mechanics, and instruments. This would also allow students to become familiar with the various equipment the observatory has to offer. The over arching wish of the faculty at the observatory is to complete a fully automated dome control, so that remote observing would become feasible. This task has been started, but is still missing a robust feedback system, some mechanical interfaces, and software interfacing with the telescope protocols. Select projects have been proposed for completion in the 2018 fall semester.  
      % The abstract is a brief summary of the design document. Typically it includes the purpose of the design document, key goals or objectives, and justifications.
      % Be sure not to confuse the abstract with the introduction.
      % It is easiest to write the abstract after the rest of the paper has been written.
      % That way you can choose key information from the sections that you've already completed and string them together in the abstract.
      % Consider the abstract to be your elevator pitch to anyone reading this design document.
      % What are they reading?
      % What is the goal?
      % Why is it worth my time?
      % The abstract is what will show up in Google results and other search engines, and what people will read when they are deciding what is worth their time and brain power.
\end{abstract}

\label{sec:nomenclature}
\newcommand{\nomunit}[1]{%
\renewcommand{\nomentryend}{\hspace*{\fill}#1}}
\renewcommand{\nompreamble}{
    % If you include mathematical expressions or express variables in the design document, list them with their corresponding definitions here as a list.
    % The two lines below make it look nice when defining units/values to constants.

    % Note that math terms and non-math terms are separated and alphabetized, regardless of the order in which they are defined. (Recall terms \$like this\$ are in the math environment)
    % Read more about advanced nomenclature formatting here:\\
    % \url{https://www.sharelatex.com/learn/Nomenclatures}
  }
\nomenclature{RIT}{Rochester Institute of Technology}
\nomenclature{SPEX}{RIT Space Exploration}
\nomenclature{MDD}{Mini Definition Document}
% Below are examples of using nomenclature for math symbols and constants or units
\printnomenclature{}


% HELPFUL HINT: If you get the warning ``Command terminated with space.'' when using a \command try placing ``%'' or ``{}'' immediately following the command.

% The sections included here are required. Additional sections and subsections may be added as necessary.
\section{Introduction}
\label{sec:introduction}
  % The introduction is a place to give background and context before diving into the subject matter.
  % Establish context for the work you are about to propose and the main ideas of the proposition itself.

\IEEEPARstart{R}{IT} SPEX has direct ties to the RIT Observatory though one of the primary advisors and a mutual interest in astronomy.  The RIT SPEX astrodynamics team started a relationship last year through projects based at the facility. The team hopes to continue this relationship, and will be continuing the rolloff dome motor project from the previous year. A new computer vision project may yield results that would be beneficial for observations, but also to the RIT Observatory as well. Designated simpler projects for incoming students are a new concept that the team will explore this semester in an effort to upkeep knowledge transfer while getting smaller tasks completed.

RIT SPEX has always had members interested in observing using various telescopes, but these members have little training. There have already been projects that gave the team insight into the equipment behind observing.  These include: building and modifying a rolling mount for its 12" Meade telescope, building mounts for an Orion sighting scope, building a barn door camera mount, and general maintenance on all telescopes. The team is looking for more projects similarly to expand their skill sets.


\section{Primary Objective}
\label{sec:primary-obj}
  % At the end of the day, whether the project ``succeeds'' or ``fails'' is judged against the objectives it sought to meet.
  % Note that results that contradict expectations/hypotheses are not failures if the scientific \& engineering methods are followed along the way.
  % Sometimes our expectations are wrong and that can be just as successful as getting data we thought we'd see.
  % What matters are what questions you intend to answer.
  % This is the main purpose or main goal the project hopes to achieve.

There are three projects selected to be worked on in the fall semester, and a new concept of mini projects.  

\subsection{Roll Off Motor Repair}
\label{subsec:rolloff}
The observatory has two domes, yet one is not very widely used because of the system to remove the roof. The current system is a hand crank which rolls the entire roof off of the structure. This is inconvenient and must be done with two people. Years ago, a motor system was installed so that the crank system would not be needed. This motor system was deemed too fast by RIT Safety and forbidden to be used. The faculty at the observatory would like to use this dome more frequently and believe this would be possible if the motor is repaired, slowed down, and a warning system installed. The motor has already been cleaned off and repaired. Numerous designs for slowing down the motor have been researched: variable frequency drive (VFD), high voltage resistors, dimmer switches, gear ratios, etc. A concrete, well documented design needs to be solidified and reviewed before proceeding to make any purchases or adjustments to the system. A warning system has been designed, but could be potentially improved and is contingent on slowing the motor speed first. More testing of the motor to characterize it is needed before a solution can be decided.

\subsection{Radio Telescope}
\label{subsec:radio}
  Radio astronomy is a common form of observation that many people may not know or consider. Observing the sky out of the optical range opens up another world of research. The team has suggested that an easier project for incoming students would be beneficial for member retention, and teach new members basic skills needed for the other projects. Last year for Imagine RIT, a simple radio telescope was constructed. Unfortunately, there is an unknown problem preventing the telescope from working properly. This project could teach basic astronomy understanding, data collection, and telescope use. It is also a great hands on demonstration for outreach. The first step of this project is to create a simple radio receiver which will read radio waves when pointed at the Sun and indicate it is receiving signal. The parts needed for this should be already acquired: low noise blockdown converter, structure, dish, satellite finder. It is a relatively low cost project, even if all new parts are needed. The team can decide to modify the structure that is already there, need be, and can continue the project once the first step is complete. Later stages would be implementing stepper motors into the structure, refining the signal measurements, and interpreting those measurements.
  
  \subsection{Star Computer Vision}
\label{subsec:star}
 When running an automated observing facility the unique problem of calibration arises. As this is the ultimate goal for the RIT Observatory, a meaningful way of confirming position is desired. Each time that a new observation session starts the telescope needs to be re-calibrated to determine where it is in the night sky. The scope will often direct the user to point to specific, or the brightest, stars that they can visibly see.  With that information it then uses system databases and GPS to determine where one is in relation to the night sky.  However, this process is tedious and it is difficult, sometimes even in a manual environment. Calibration can be done multiple times a night, if a system starts to drift from the original target as it moves through the night. The solution to this problem is to have some sort of system that knows directly about the night sky within the field of view of a spotting scope to then help direct the telescope to objects/stars of interest for calibration.  Using computer vision and a process called plate solving this becomes possible.
The premise of plate solving is that you get unique information about the stars in a field of view from some image, or video stream.  This data is often x/y coordinates, details about the field of view, as well as magnitude which can all be obtained by computer vision post-processing of an image.  Then those features are compared to others in an indexed dataset.  It realistically becomes a large computer science search problem as one is searching a larger database for where their feature points occur.
In addition this is also helpful for future research if a student takes an image of the night sky and are unaware of what they photographed.  They can simply input the image into the software and it would be able to tell them the major stars and points of interest.

  \subsection{Mini Research Projects}
\label{subsec:mini}
 Historically, the astrodynamics team has worked on more engineering based projects. There has been interest in more physics and math based learning projects for members to partake in. The "mini projects" are questions that have been previous worked on or suggested, but are worked on individually by members of the group on their own time. Each mini project starts with a mini definition document (MDD). This is a one page document which outlines the problem, gives a little background, and suggests resources to get started.  These do not have a time-line, but every week at the meeting time will be given for individuals to discuss their process with each other, and seek guidance from older members who have completed these questions before. Topics of these include: orbital trajectory simulations, astronomy coordinate system conversions, spherical geometry, etc.

\subsection{Observatory Modifications}
\label{subsec:obs}

Along the same lines of the mini projects, these are smaller, simplified modifications desired by the RIT Observatory staff that do not warrant a full PDD or team. These tasks can be taken on by a member or two from the team and completed without a solid time-line. This gives students an opportunity to interface with the observatory more, and develop basic skills in the engineering realm with simple projects. Tasks include: desiging a safety for the CCD, mounting an LED light for images known as "flats", and updating the RIT Observatory website.

  
  

\section{Benefit to SPEX}
\label{sec:benefit}
% One of the core values of SPEX is to provide opportunities for academic and professional growth for its members,
% and to challenge them with interesting projects.
% In this section, explain how the project would benefit SPEX members as students,
% space enthusiasts, and young professionals.
By working with the equipment and the facility, SPEX students will have access to the observatory and gain experience with the equipment. Since SPEX is an organization for all students of all majors, students with an interest in astronomy and telescopes who may not have the opportunity to take astronomy classes will have access to the observatory. Ideally, this collaboration benefits SPEX, the observatory, and all students at RIT.


% Below I have used subsections to identify key ideas in this section. These particular subsections are not required as part of the SPEX Standard, but serve as an example of using subsections in a text.

\subsection{Engineering Skills}
\label{subsec:mindset}
Firstly, these projects will help students develop key engineering skills, as do most projects that SPEX takes on. These projects will take students through the engineering process from beginning to end, involving problem solving, slight planning, research, and implementation. Students will have the opportunity to be involved in every step of the way if desired. Some higher level engineering skills may be required to complete some of these projects, but other provide students with the chance to learn or enhance their experience with Arduino, circuits, hand tools, power tools, integration, and control systems. 

\subsection{Member Retention}
\label{subsec:traceability}
Secondly, while working with the equipment at the observatory and other astronomy related projects, students will become more familiar with the workings of the telescopes. This will help students understand what happens behind taking observations, and directly effect their observational skills with the SPEX telescopes. They will become more comfortable with the equipment itself.
\subsection{Member Interaction}
\label{subsec:plug-n-play}
SPEX is an organization for all students of all majors; students with an interest in astronomy and telescopes who may not have the opportunity to take astronomy classes will have access to the observatory. If steps towards remote observing are taken and eventually implemented, it will lighten the burden of observing on the faculty, which in turn opens the observatory to more students. A strong bond with the observatory also opens up more opportunities there for SPEX students specifically. 

\subsection{Diversity of Learning}
\label{subsec:diverse}

The diverse projects touch on all different aspects of astrodynamics and touch on multiple disciplines. The rolloff motor deals with electrical systems and control systems of telescope facilities. Computer star vision pertains to computer science and data processing. The radio telescope touches on optics and astronomy basics. The various MDD's typically define physics and math based projects. Students have the opportunity to learn skills from various disciplines pertaining to the broad topic of astrodynamics. 


\section{Implementation}
\label{sec:implementation}
  % What path do you anticipate the project to take?

The projects can be worked on simultaneously, or one after the other depending on the number of interested members. Ideally, projects that require more work in the outdoors would want to be completed first to avoid weather concerns. If these main first projects are implemented, there are more projects that can be started in a new PDD, or these can be enhanced and continued in the spring. Being the first semester of implementing freshman learning projects, the goal is to gain feedback throughout and adjust the process accordingly. All mini projects and freshman projects will have a one to two page beginning document, outlining the objective, background, and resources needed to complete the project. These do not have a time limit yet, but ideally these projects do not take a full semester to complete. Documentation of our process and knowledge sharing are important requirements of this year. The one page "mini PDD" is the first implementation of this concept.



\subsection{Deliverables}
\label{subsec:deliverables}
  % When all is said and done, what will you have to show for it?
  % Examples: Hardware, software, poster, ImagineRIT demo, presentations, technical papers...
Physical deliverables are not always the outcome of these projects, unlike in the past. 

By the end of the semester the roll off motor should have a thoroughly reviewed and designed system neatly documented. The system could be implemented in the spring. 

The final deliverable in the project is software running on a computer that hooks up to a webcam, with a fixed aspect and zoom, that can detect stars and get positional data from them.  That positional data can then be fed into an algorithm to identify the feature points in the night sky for calibration.  Scope is limited to just the development of the algorithm on a specific device, with the potential to configure additional parameters for other devices.  This is because there are two types of plate solving algorithms.  One is agnostic to the camera and is more complicated than the other which has specific details about the camera.  It becomes complicated because different cameras have different fields of view and the x/y coordinates can be offset from what's in the database.  Eventually, once a working prototype is in place, the group can have a follow on effort to make it platform agnostic if that is desired functionality.

The mini projects vary, but are physics and software based so they do not have a physical deliverable. The deliverable will usually be a program that calculates trajectories, orbits, or coordinates.


\subsection{Milestones}
\label{subsec:milestones}
  % Be as detailed as you can, but it's okay if there are unknowns.
  % At the very least, specify how many semester you expect the project to take until it reaches completion.

There are no specific time-lines needed for any of the mini projects and observatory modifications, although they are not suspected to take an entire semester. Each week at the team meeting, 15 minutes will be given to check up on each individual student's progress. This period will allow the students to ask their questions, as well as allow the older students to give guidance. A small review will be held week 5 of their findings and progress.

The radio telescope project geared towards freshman will also be set up similarly. Since troubleshooting the original problem is not expected to take more than 3 weeks, a schedule will be tailored once concrete new modifications are solidified. The team will get the rest of the meeting not used for the mini project check ins and general updates to work on this and seek guidance from older members. A review will be held week 5 of their findings and modifications.

The other two main projects have time-lines outlined in Tables \ref{tab:computervis} and \ref{tab:rolloff}. The time-lines depend on the number of people available to work on the projects.

\begin{table}[h!]
    % the "h" in these brackets tells LaTeX to put the table Here. Try [t] for top and [b] for bottom,
    % or [hbp] for "here, or if you can't do that put it at the bottom of the page, or if you can't do that put it on its own page.
    % Here we've also used an "!" to yell at LaTeX to DO THIS OR ELSE!
    \caption{Notional Time Line of Project Milestones - Star Computer Vision}
    \centering
    \begin{tabular}{@{}cll@{}}
    % the letters here ^^^^ designate the columns.
    % (l=left align, c=center, r=right align)
    % the weird @{} thingies tell LaTeX to not have left-right padding between cells
    % so cells butt up right against the edge
    \toprule
    Phase & Task & Duration \\
    \midrule
    1 & Research plate solver algorithms & 3-4 weeks\\
    & and other solutions \\
    2 & Writing of algorithm, handling & 5-6 weeks \\ 
    & all math, and getting system \\
    & working on a webcam \\
    3 & Capturing a dataset of test images & 3 weeks \\
    & and verifying the algorithm works\\
    & as expected \\
    4 & Generate documentation and delivery to SPEX & 1 week  \\
    \bottomrule
    \end{tabular}
\label{tab:computervis}
\end{table}

\begin{table}[h!]
    % the "h" in these brackets tells LaTeX to put the table Here. Try [t] for top and [b] for bottom,
    % or [hbp] for "here, or if you can't do that put it at the bottom of the page, or if you can't do that put it on its own page.
    % Here we've also used an "!" to yell at LaTeX to DO THIS OR ELSE!
    \caption{Notional Time Line of Project Milestones - Rolloff Motor}
    \centering
    \begin{tabular}{@{}cll@{}}
    % the letters here ^^^^ designate the columns.
    % (l=left align, c=center, r=right align)
    % the weird @{} thingies tell LaTeX to not have left-right padding between cells
    % so cells butt up right against the edge
    \toprule
    Phase & Task & Duration \\
    \midrule
    1 & Testing, characterization  & 3 weeks\\
    & of motor and document collection  \\
    2 & Research of different methods & 3 weeks \\ 
    & to modify the motor \\
    3 & Design entire system and  & 3 weeks \\
    & document\\
    4 & Design review with applicable & 1 week  \\
    & parties \\
    5 & Redesign based on feedback & 3 weeks \\
    6 & Generate documentation and delivery to SPEX & 1 week  \\
    \bottomrule
    \end{tabular}
\label{tab:rolloff}
\end{table}
\section{Externalities}
  % Things not directly related to the work or outcomes, but related to the project as a whole.
\subsection{Prerequisite Skills}
  % Which skills do team members need to have before work can start (not including skills that will be learned ``on the job'')?
  Leadership with the prerequisite skills has been determined. It would be beneficial for some students to already have basic engineering and hand tool skills. Knowledge of Arduino code and basic electrical circuits is also beneficial. These skills do not need to be present in all group members as these projects are an opportunity for new members to learn these skills. For the roll off motor repair, someone with electrical engineering or motor  knowledge has been deemed crucial. Recruitment is a high priority for this situation. 
  
\subsection{Faculty Support}  
  Direct support from Dr. Richmond and Dr. Connelly is
expected. Access to the facility is highly dependent on one
of the above being present. Both faculty are very familiar
with the equipment, telescopes, and facility. They also have
experience with what is typical for systems at other observatories, which may be beneficial when working on these
projects. Dr. Richmond has been modifying equipment at
the observatory for decades. Dr. Connelly understands the
structure and capabilities of RIT SPEX through being an
advisor. Both are great resources for guidance during the
completion of the projects. As all professors, they are juggling
multiple responsibilities at one time, which can complicate
scheduling time for this project. Even so, they understand the
needs of the observatory and can also be seen as the primary
customers for the roll off motor and Observatory modification projects, so it is important to keep them
informed through the engineering process. They will be signing off on any implementations at the Observatory beforehand.
  
\subsection{Funding Requirements}
  % Estimate costs that would be needed to meet objectives.
  Primary funding for these projects is used for the rolloff motor, which will come from the Observatory itself. Extra funds have been requested from SG, due to their developed SG club status. Small amounts of funding may be requested through the SPEX budget as a last resort. Table \ref{tab:fund} outlines the estimated costs of the projects. The costs have been overestimated for planning purposes.
\begin{table}[h!]
  
    \caption{Notional Cost of Projects}
    \centering
    \begin{tabular}{@{}cll@{}}
    % the letters here ^^^^ designate the columns.
    % (l=left align, c=center, r=right align)
    % the weird @{} thingies tell LaTeX to not have left-right padding between cells
    % so cells butt up right against the edge
    \toprule
   Project & Cost \\
    \midrule
    Roll Off Motor Repair\IEEEauthorrefmark{1}   &  500 \\
    Radio Telescope & 100 (max) \\
    Observatory Modifications & 100\\
    Mini Projects & 0\\
     Star Computer Vision & 0\\
    \bottomrule
    \IEEEauthorrefmark{1} The cost of the motor repair may be generously high because it \\ 
greatly depends on the condition of the current motor
    \end{tabular}
\label{tab:fund}
\end{table}

\subsection{Long-Term Vision}
\label{sec:vision}
The long term goal of these projects is to work towards a fully automated dome system. This would enable remote observing for the Observatory and enable increased access to more students.
  Students will get familiar with the technology behind observing and enhance their basic engineering skills. As projects get completed, a strong bond between the RIT Observatory and RIT SPEX can be continued.

\section*{Acknowledgements}
The author would like to thank the advisors of SPEX, the faculty at the
observatory for reaching out about these projects, the team members of 2018-2019 that worked extremely hard and the
SPEX admins for their extra effort put in to make this project
process possible.



\end{document}
